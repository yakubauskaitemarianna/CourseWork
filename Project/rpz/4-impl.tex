\chapter{ Технологический раздел}
\label{cha:design}

Разработка встраемовой в графические редакторы библиотеки, с целью автоматизации методов закраски графических новелл, в частности, манги.


\section{Выбор инструментов для разработки.}
Так как создание библиотеки с целью автоматизации методов закраски графических новелл подразумевает применение методов статистики для анализа данных, работу с нейронными сетями, выбор инструмента разработки был остановлен на языке программирования Python 3.7. Python является высокоуровневым интерпритируемым и кросплатформленным языком программирования. Для глубокого машинного обучения использовалась библиотека Chainer, поддерживающая работу со сверточными нейросетями, а так же позволяющая использовать ускорение вычислений на GPU - с помощью набора инструментов от CUDA. Библиотека использует систему многоуровневых узлов, которая позволяет вам быстро настраивать, обучать и развертывать искусственные нейронные сети с большими наборами данных. 

\section{Процесс обучения.}
Процесс обучения подразумевает собой подготовку data-set $[nico-opendata]$, для решения задачи колоризации), инициализацию функции потери, использующуюся на этапе классификации и подготовку data-set  $[manga_109]$ для реализации соотвествия между именем и персонажем.


%%% Local Variables:
%%% mode: latex
%%% TeX-master: "rpz"
%%% End:
