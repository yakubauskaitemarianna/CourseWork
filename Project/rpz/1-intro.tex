\Introduction

Японские комиксы, называемые мангой популярны во всем мире. Традиционно
они создаются в черно-белом стиле, а колоризацией готовой манги, ее раскраской,
занимается группа художников. Вместо классических способов передачи теней
используется штриховка или скринтон (см. аналитический раздел). Вот почему колоризация
манги требует детального разбора каждого кадра. Способ автоматизации
процесса анализа изображения и дальнейшая закраска будут полезны при создании
цветной манги, выступающей в роли сценария к аниме.
Автоматизированная обработка графических новелл - это проблема, стоящая
на стыке компьютерной графики и машинного зрения. Сложность задачи колоризации
манги заключается в работе с монохромным изображением, в котором
использованы различные виды штриховки и типы линий для передачи текстур. Аниме
(японские мультфильмы, снимаемые по сюжету манги), отличаются закраской:
используется многогранная заливка и размытие по краям, для придания эффекта
рисовки пастелью или акварелью, и создания более мягких и сложных оттенков. .
Программа, цель работы которой сводится к автоматизации процесса колоризации,
ускорит процесс работы над мультипликационным фильмом.
Глобально рассматриваются следующие варианты решения проблемы:

\begin{enumerate}
\item Закраска черно-белого рисунка, используя шаблон.
\item Закраска с использованием расставленных цветовых акцентов
\end{enumerate}

Анализ методов, выбор оптимального решения проблемы и его реализация -
цель курсового проекта.
 
  
 
 
 
